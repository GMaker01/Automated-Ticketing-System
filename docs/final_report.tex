\documentclass{paper}
\usepackage{multicol}
\usepackage{fullpage}
\usepackage{graphicx}
\usepackage{caption}
\usepackage{subcaption}
\usepackage{float}
\usepackage{multirow}
\usepackage[hyphens]{url}
\graphicspath{ {./images/} }

\title{Automated Ticketing System}
\author{Anthony Satuner, Abhishek Burli, Geetha Charan Reddy}


\begin{document}
    \maketitle
    \begin{abstract}
        One  of  the  key  activities  of  any  IT  function  is  to  “Keep  the  lights on” to ensure there is no impact to the Business operations. IT leverages Incident Management process to achieve the above Objective. These Incidents are recorded in form of tickets and are assigned to certain support teams inorder to resolve them. However the time taken to Analyse, Assign and Re-Assign is really large and using ML and DL techniques this process could be automated, reducing FTE to focus on more critical issues. This report is an analysis perfomed on the following situation using a simulated dataset. An effective DL model was built which yielded an accuracy of $\sim$92 \% accuracy on unseen data. It was also observed the issues faced in the dataset are majorly generated by job\_scheduler error which can be solved by perfoming RCA (Root Cause Analysis) and automating Password-Reset process. Hence a cumulative  Resource / FTE  allocation  reduction  by  approximately  37.34\% is possible and Business can operate at $\sim$62.66\% of original estimates.
    \end{abstract}
    
    \begin{keywords}
        ML, DL, doc2vec, embeddings, LDA, FTE, SLA, Mojibake, Ticket Assignment, Incident Management Process, NLP
    \end{keywords}

    \input{docs/Introduction.txt}
    \input{docs/Dataset.txt}
    \input{docs/Exploratory Analysis.txt}
    \input{docs/feature engineering.txt}
    \input{docs/Modelling.txt}
    \input{docs/Conclusion.txt}

    \nocite{*}
    \newpage
    \bibliographystyle{ieeetr}
    \begin{multicols}{2}
    \bibliography{citations}
    \end{multicols}
\end{document}